\documentclass[20]{jarticle}
\usepackage{ketpic,ketlayer}
\usepackage{amsmath,amssymb}
\usepackage{graphicx}
\usepackage{color}
\usepackage{emath}

\setmargin{20}{20}{20}{20}

\begin{document}

\begin{itemize}
\item タイトル\\
複素数の和と複素数平面
\item 目的\\
この教材は,複素数の和を複素数平面上で図形的に捉えることを
%認識することを
%理解することを
目的とします.
\item 内容\\
%この教材では,
複素数平面上で,複素数を原点とその複素数に対応する点を結んでできる矢印として%図形的に
表すことにします.\\
「出題」のボタンを押すたびに,2つの複素数$\alpha,\ \beta$が与えられ,複素数平面上には$\alpha,\ \beta$を
表す黒い矢印と$\alpha+\beta$を表す赤い矢印が与えられます.$\alpha+\beta$に対応する点
を%移動させて赤い矢印を
正しい位置に動かすと「正解」の言葉とともに複素数の和を表す図形が完成します.
\item 使い方\\
次の手順で操作します.
\begin{enumerate}[(1)]
\item 名前や番号を「ID欄」に入力する.
\item 「出題」のボタンを押すと,2つの複素数$\alpha,\ \beta$が与えられる.同時に複素数平面上には
$\alpha,\ \beta$を表す黒い矢印と$\alpha+\beta$を表す赤い矢印が与えられる.
\item $\alpha+\beta$に対応する点を正しい位置に動かし,「採点」のボタンを押す.
\item 赤い点の位置が正しければ「正解」,正しくなければ「不正解」の文字が表示される.
\item 「正解」が表示されるまで点を動かして「採点」のボタンを押してよい.
\item 「正解」が表示されると,%矢印と点線で囲まれた
複素数の和を表す図形が完成する.
\item 「出題」のボタンを押し,$\alpha$と$\beta$の値を変更し, 他の複素数で上の操作を適宜
繰り返してよい.
\item 複素数の和を図形的に捉えることができたら「記録」ボタンを押し終了する.

\end{enumerate}

\end{itemize}
\end{document}
